%!TEX TS-program = xelatex
\documentclass{nihbiosketch}

%------------------------------------------------------------------------------

\name{Brokamp, Cole}
\eracommons{brokampr}
\position{Research Fellow}

\begin{document}
%------------------------------------------------------------------------------

\begin{education}
University of Cincinnati  & B.S           & 06/2010  & Biomedical Engineering \\
University of Cincinnati               & Ph.D.         & 04/2016  & Biostatistics and Bioinformatics \\
Cincinnati Children's Hospital Medical Center  & Postdoctoral Research Fellow  & present  & Biostatistics and Epidemiology \\
\end{education}


\section{Personal Statement}

\begin{statement}
As a biostatistician, I have specialized myself in the areas of machine learning and its application to large environmental datasets.  These types of datasets are similar to ``-omics'' datasets in that the number of variables often outnumbers the number of observations and these variables are usually highly correlated. I will use my previous experience with simulating datasets and comparing statistical methodologies to contribute to Dr. Ambroggio's specific goal of identifying the best missing data imputation strategy to use in metabolomics data analysis. \\

\noindent Related Publications:

\begin{enumerate}

\item \textbf{Cole Brokamp}, MB Rao, Patrick Ryan, Roman Jandarov. A comparison of resampling and recursive partitioning methods in random forest for estimating the asymptotic variance using the
infinitesimal jackknife. arXiv preprint arXiv:1706.06150.  2017.

\item \textbf{Cole Brokamp}, Roman Jandarov, MB Rao, Grace LeMasters, Patrick Ryan. Exposure assessment models for elemental components of particulate matter in an urban environment: A comparison of regression and random forest approaches. Atmospheric Environment. 151. 1-11. 2017.

\item \textbf{Cole Brokamp}, MB Rao, Tina Zhihua Fan, Patrick H Ryan. Does the elemental composition of indoor and outdoor PM2.5 accurately represent the elemental composition of personal PM2.5? Atmospheric Environment. 101:226-234, 2015.

\end{enumerate}

\end{statement}

%------------------------------------------------------------------------------
\section{Positions and Honors}

\subsection*{Positions and Employment}
\begin{datetbl}
2010--2016  & Graduate Research Assistant, University of Cincinnati\\
2016--  & Research Fellow, Cincinnati Children's Hospital Medical Center\\
\end{datetbl}

\subsection*{Honors}
\begin{datetbl}
2010            & B.S. awarded with Distinguished Honors, University of Cincinnati\\
2016            & CCHMC Division of Biostatistics \& Epidemiology Travel Award\\
2016            & CCHMC Arnold W. Strauss Fellowship Award\\
\end{datetbl}

%------------------------------------------------------------------------------

\section{Contribution to Science}

I recently applied machine learning methods to estimate the exposure of children to different types of air pollution and quantified its effects on their respiratory and mental health. Specifically, I created a novel land use random forest model (LURF) and showed that it was more accurate and precise than traditional land use regression models. This work has been implemented into an R package that helps researchers to extract statistical inferences from random forests, as well as to aid users in the application and visualization of big spatial data.\\

\noindent Related Publications:

\begin{enumerate}
	
\item \textbf{Cole Brokamp}, Roman Jandarov, MB Rao, Grace LeMasters, Patrick Ryan. Exposure assessment models for elemental components of particulate matter in an urban environment: A comparison of regression and random forest approaches. Atmospheric Environment. 151. 1-11. 2017.

\item Kelly J Brunst, Patrick H Ryan, \textbf{Cole Brokamp}, David Bernstein, Tiina Reponen, James Lockey, Gurjit K Khurana Hershey, Linda Levin, Sergey A Grinshpun, Grace LeMasters. Timing and duration of traffic-related air pollution exposure and the risk for childhood wheeze and asthma. American Journal of Respiratory and Critical Care Medicine. 192(4). 421-427. 2015.

\item Jennifer Kannan, \textbf{Cole Brokamp}, David I. Bernstein, Grace K. LeMasters, Gurjit K. Khurana Hershey, Manuel Villareal, James E. Lockey, Patrick Ryan. Parental Snoring and Environmental Pollutants, but Not Aeroallergen Sensitization, Are Associated with Childhood Snoring in a Birth Cohort. Pediatric Allergy, Immunology, and Pulmonology. 0. 2016.

\end{enumerate}

\noindent Related Software:

\begin{enumerate}

\item Cole Brokamp. (2016, November 3). cole-brokamp/aiRpollution 0.2. Zenodo. \url{http://doi.org/10.5281/zenodo.164697}.

\item Cole Brokamp. (2016, June 10). cole-brokamp/RFinfer 0.2. Zenodo. \url{http://doi.org/10.5281/zenodo.50879}.

\end{enumerate}


\subsection*{Complete List of Published Work in MyBibliography:} 
\url{https://www.ncbi.nlm.nih.gov/myncbi/browse/collection/49821426}


%------------------------------------------------------------------------------

\section{Research Support}

\subsection*{Ongoing Research Support}

\bigskip

\textbf{NIH/NIEHS 1R01ES019890-01}\\
\emph{Neurobehavioral and Neuroimaging Effects of Traffic Exposure in
	Children}\\
Ryan, PI (7/1/12 - 3/31/18)\\
The association between exposure to traffic-related air pollutants
(TRAP) during early childhood and neurobehavioral and neuroimaging
outcomes has not been thoroughly examined. The objective of the proposed
study is to determine if children exposed to increased levels of TRAP
during critical time periods of brain development have altered
neurobehavior in childhood as measured by a battery of valid and
reliable tests and to assess the physiologic impact of TRAP exposure on
brain structure, organization, and function using quantitative magnetic
resonance imaging (MRI). These results will fill important gaps in
current scientific knowledge related to the relationship between TRAP
exposure and neurobehavior and central nervous system effects.\\
Role: Biostatistician

\bigskip

\textbf{NIH 5K23AI121325}\\
\emph{Biomarkers and Risk Stratification in Pediatric Community}\\
Florin, PI (01/01/16 - 12/31/19)\\
The extensive variation in care, in addition to the lack of
evidence-based decision aids, highlights the critical need for an
improved understanding of disease severity and tools to guide management
for pediatric CAP. The proposed research will address this important
knowledge and practice gap.\\
Role: Biostatistician

\bigskip

\textbf{U01HG008666}\\
\emph{EMERGE: Better Outcomes for Children: Promoting Excellence in
	Healthcare Genomics to Inform Policy}\\
Harley, PI (09/01/15 - 05/31/19)\\
We have developed algorithms for the electronic health record (EHR), led
the Pediatric Workgroup, developed pharmacogenomics, evaluated the
preferences of parents and caregivers to advance genomic medicine and
assimilated technical advances into our EHR. The eMERGE effort has
become the basic fabric of the institutional initiative to incorporate
the extraordinary advances of genetics, genomics and the electronic
medical record into healthcare.\\
Role: Biostatistician

\bigskip

\textbf{Internal ARC - Cincinnati Children's Hospital}\\
\emph{Mother Infant Data Hub}\\
Marsolo, PI (7/1/15 - 7/1/18)\\
The goals of this award are to create a research database of
comprehensive clinical coverage for neonates born throughout the greater
Cincinnati area including linkage of medical records to external data
sets at the individual- and area-level during the first year of life.\\
Role: Biostatistician

\bigskip

\textbf{Internal ARC - Cincinnati Children's Hospital}\\
\emph{CARPE DIEM}\\
Ambroggio, PI (7/1/15 - 7/1/18)\\
The goals of this award are to develop a diagnostic tool based on the
urinary metabolome that can differentiate between viral and bacterial
community-acquired pneunomia in children.\\
Role: Biostatistician

\bigskip

\textbf{Internal - University of Cincinnati}\\
\emph{Epidemiology of Rural/Urban Disparities in Stroke}\\
Jasne, PI (1/1/17 - 12/31/17)\\
The goal of this project is to identify stroke incidence disparities
among rural and urban geographic areas.\\
Role: Biostatistician

%------------------------------------------------------------------------------

\subsection*{Recently Completed Research Support}

\bigskip

\textbf{Internal Processes and Methods Award - Center for Clinical \&
	Translational Science \& Training}\\
\emph{Validating a Geocoding Approach for Multi Site Studies}\\
Brokamp, PI (1/24/17 - 6/30/17)\\
The primary objective of this award is to compare the geocoding
(assigning latitude and longitude coordinates to addresses) accuracy of
our software DeGAUSS (DEcentralized Geomarker Assessment for mUlti Site
Studies) to with other common geocoding software. Furthermore, each
method will be evaluated based on it ability to correctly estimate
environmental exposures and community-level characteristics.\\
Role: PI

\bigskip

\textbf{Internal Arnold W. Strauss Fellowship Award - Cincinnati
	Children's Hospital}\\
\emph{Assessing Exposure to Air Pollution Across Time and Space}\\
Brokamp, PI (7/1/16 - 6/30/17)\\
The primary objective of this award is to combine satellite-based
measurements, land use characteristics, and meteorologic data to create
a hybrid spatiotemporal model for ground level exposure to particulate
matter using exact addresses and dates.\\
Role: PI

\bigskip

\textbf{HEI 4784-RFA08-1/09-5}\\
\emph{Analysis of Personal and Home Characteristics Associated with the
	Elemental Composition of PM2.5 in Indoor, Outdoor, and Personal Air in
	the RIOPA Study}\\
Ryan, PI (12/1/12 - 11/30/13)\\
The purpose of this study is to assess the relationship between
concurrent measurements of the elemental composition of PM2.5 in indoor,
outdoor, and ambient air and the elemental composition of indoor,
outdoor and personal air across individuals and cities. The study will
also identify personal, home, and environmental factors significantly
associated with specific elements or clusters of elements in PM2.5.\\
Role: Biostatistician

\bigskip

\textbf{Academy Health}\\
\emph{Community Health Peer Learning Program: Participant Community}\\
Beck, PI (2/1/16 - 6/30/17)\\
The goals of this project are to reduce by 10\% the inpatient bed-day
rate for one high risk neighborhood in Cincinnati through interventions
promoted by shared data and improved data visualization.\\
Role: Biostatistician



\end{document}

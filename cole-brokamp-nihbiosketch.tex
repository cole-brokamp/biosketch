%!TEX TS-program = xelatex
\documentclass{nihbiosketch}

%------------------------------------------------------------------------------

\name{Brokamp, Cole}
\eracommons{brokampr}
\position{Assistant Professor}

\begin{document}
%------------------------------------------------------------------------------

\begin{education}
University of Cincinnati  & B.S           & 06/2010  & Biomedical Engineering \\
University of Cincinnati               & Ph.D.         & 04/2016  & Biostatistics and Bioinformatics \\
Cincinnati Children's Hospital Medical Center  & Postdoctoral Research Fellow  & 10/2017  & Biostatistics and Epidemiology \\
\end{education}

\section{Personal Statement}

\begin{statement}

As a biostatistician and geoinformatician, I have specialized myself in the
areas of machine learning and its application to large environmental and
clinical datasets. Recent democratization of “big spatial data” and advances in
geoinformatics have allowed unprecedented access to environmental and
socioeconomic characteristics that vary highly with respect to both time and
space. More precise environmental features require more complex modeling and I
have dedicated my early career to furthering exposure science methodology in
order to bring more precise exposure assessment tools to environmental and
population health
studies. Furthermore, I have developed a novel informatics approach and
accompanying software package called DeGAUSS which allows for user-friendly
attachment of geospatial variables to existing research cohorts while
maintaining the confidentiality of private health information.

I look forward to collaborating with Drs. King, Khoury, and Szczesniak, as well
as the other co-investigators in order to establish a Biostatistics Research
Center to facilitate the understanding of glycemic profiles during pregnancy.
I will utilize my experience and expertise in automated and reproducible geomarker
assessment in order to supplement collected longitudinal data with environmental
exposures and community characteristics to help elucidate the onset and
evolution of dysglycemia in pregnant mothers.\\

\begin{enumerate}

	\item \textbf{Cole Brokamp}, Chris Wolfe, Todd Lingren, John Harley, Patrick Ryan. Decentralized and Reproducible Geocoding and Characterization of Community and Environmental Exposures for Multi-Site Studies. \textit{Journal of American Medical Informatics Association.} 25(3). 309-314. 2018.
	
	\item \textbf{Cole Brokamp}. DeGAUSS: Decentralized Geomarker Assessment for Multi-Site Studies. \textit{Journal of Open Source Software}. 2018. 

\end{enumerate}

\end{statement}

%------------------------------------------------------------------------------
\section{Positions and Honors}

\subsection*{Positions and Employment}
\begin{datetbl}
2012--2016 & Research Associate, Department of Environmental Health, University of Cincinnati \\	
2016--2017  & Research Fellow, Cincinnati Children's Hospital Medical Center Division of Biostatistics \& Epidemiology\\
2017--      & Assistant Professor of Pediatrics, the University of Cincinnati Department of Pediatrics and Cincinnati Children’s Hospital Medical Center Division of Biostatistics \& Epidemiology\\
\end{datetbl}

\subsection*{Honors}
\begin{datetbl}
2010            & B.S. awarded with Distinguished Honors, University of Cincinnati\\
2016            & CCHMC Division of Biostatistics \& Epidemiology Travel Award\\
2016            & CCHMC Arnold W. Strauss Fellowship Award\\
2017            & CCHMC Epidemiology \& Biostatistics Top Publication \\
2017            & CCHMC Epidemiology \& Biostatistics Top Research Achievement \\
\end{datetbl}

%------------------------------------------------------------------------------

\section{Contribution to Science}

\begin{enumerate}


\item The main aim of my early career work has been to develop exposure assessment models for airborne pollutants based on machine learning techniques.  Specifically, applying random forest to land use models results in higher accuracy and precision of air pollution exposure assessment by elucidating complex interactions and nonlinear relationships between land use predictors and pollutant concentrations. This work includes the first machine learning or ensemble model used to assess exposure to elemental components of particulate matter. Recent introduction of remote sensing satellite data has allowed for extension of the land use random forest model to produce daily estimates of air pollution from 2000 - 2015 at a resolution of 1 x 1 km across the Greater Cincinnati area.

\begin{enumerate}
	
	\item \textbf{Cole Brokamp}, Roman Jandarov, Monir Hossain, Patrick Ryan. Predicting Daily Urban Fine Particulate Matter Concentrations Using Random Forest. Environmental Science \& Technology. 52 (7). 4173-4179. 2018.

	\item \textbf{Cole Brokamp}, Roman Jandarov, MB Rao, Grace LeMasters, Patrick Ryan. Exposure assessment models for elemental components of particulate matter in an urban environment: A comparison of regression and random forest approaches. \textit{Atmospheric Environment}. 151. 1-11. 2017.
	
	\item \textbf{Cole Brokamp}, MB Rao, Patrick Ryan, Roman Jandarov. A comparison of resampling and recursive partitioning methods in random forest for estimating the asymptotic variance using the infinitesimal jackknife. \textit{Stat}. 6(1). 360-372. 2017.

\end{enumerate}

\item Collaborating with other researchers wishing to use geospatial characteristics of research participants within existing cohorts and multi-site studies has highlighted the significant need for a reproducible and distributed method for extracting place-based information from a residential address while maintaining the privacy of protected health information. I recently developed a novel approach and accompanying software package called DeGAUSS which overcomes the multiple challenges in the use of address data in multi-site studies and also serves as a more general reproducible research tool for geocoding and geomarker assessment. This approach is currently being implemented in a wide variety of national environmental health studies.

\begin{enumerate}
	
	\item \textbf{Cole Brokamp}, Chris Wolfe, Todd Lingren, John Harley, Patrick Ryan. Decentralized and Reproducible Geocoding and Characterization of Community and Environmental Exposures for Multi-Site Studies. \textit{Journal of American Medical Informatics Association.} 25(3). 309-314. 2018.
	
	\item \textbf{Cole Brokamp}. DeGAUSS: Decentralized Geomarker Assessment for Multi-Site Studies. \textit{Journal of Open Source Software}. 2018. 
	
\end{enumerate}

\item The work in developing land use random forest methods has helped me gain expertise in geospatial computing.  Along with other collaborators, I have applied my geospatial computing and geoinformatics expertise to analyze the effect on health of environmental exposures such as combined sewer overflows, elemental components of particulate matter, community deprivation, greenspace, and ozone.

\begin{enumerate}

 \item Juliana Madzia, Patrick Ryan, Kimberly Yolton, Zana Percy, Nick Newman, Grace
LeMasters, \textbf{Cole Brokamp}. Residential Greenspace Is Associated with Childhood
Behavioral Outcomes. \textit{Journal of Pediatrics}. 2018.	\textit{In Press}.

	\item \textbf{Cole Brokamp}, Andrew F. Beck, Louis Muglia, Patrick Ryan. Combined Sewer Overflow Events and Childhood Emergency Department Visits: A Case-Crossover Study. \textit{Science of the Total Environment}. 607-608. 1180-1187. 2017.
	
	\item Lusine Yaghjyan, R Aroa, \textbf{Cole Brokamp}, E O'Meara, B Sprague, G Ghita, Patrick Ryan. Association of air pollution with mammographic breast density in the Breast Cancer Surveillance Consortium. \textit{Breast Cancer Research}. 19:36. 1-10. 2017.
	
	\item Kelly J Brunst, Patrick H Ryan, \textbf{Cole Brokamp}, David Bernstein, Tiina Reponen, James Lockey, Gurjit K Khurana Hershey, Linda Levin, Sergey A Grinshpun, Grace LeMasters. Timing and duration of traffic-related air pollution exposure and the risk for childhood wheeze and asthma. \textit{American Journal of Respiratory and Critical Care Medicine}. 192(4). 421-427. 2015.
	
	
\end{enumerate}


\item I have also contributed to several studies on the disparities of health outcomes within children and the contribution of the place-based and social determinants of health to these disparities in order to identify root causes and meaningful solutions.

\begin{enumerate}

	\item \textbf{Cole Brokamp}, Andrew F. Beck, Neera K. Goyal, Patrick Ryan, James M. Greenberg, Eric S. Hall. Material Community Deprivation and Hospital Utilization During the First Year of Life: An Urban Population-Based Cohort Study. Journal of Pediatrics. \textit{In Press.}
	
	\item Andrew F. Beck, Carley L. Riley, Stuart Taylor, \textbf{Cole Brokamp}, Robert S. Kahn. Toward a Culture of Health in Hospitals: Pervasive population disparities in inpatient bed-day rates across conditions and subspecialties. Health Affairs. 37(4). 551-559. 2018.
		
	\item Lauren C. Riney, \textbf{Cole Brokamp}, Andrew F. Beck, Wendy Pomerantz, Hamilton Schwartz, Todd A. Florin. Emergency Medical Services Utilization is Associated with Community Deprivation in Children. Prehospital Emergency Care. Online ahead of print. 2018.


\end{enumerate}

\item Lastly, I have contributed to a research team that has recently used functional data analysis combined with joint modeling (FD-JM) to identify and predict rapid decline in lung function among patients with cystic fibrosis (CF) lung disease. Translating this predictive model into an interactive application has allowed for patients and clinicians to take advantage of it at the bedside.  Focus groups and partnerships with clinicians have allowed us to iteratively develop the application based on end-user feedback. Work with the CF Foundation Patient Registry (CFFPR) to implement these models and visualizations into clinical settings has improved prognostic care.

\begin{enumerate}
	
	\item Rhonda D. Szczesniak, \textbf{Cole Brokamp}, Weiji Su, Gary L. McPhail,
    John Pestian, and John P. Clancy. Improving Detection of Rapid Cystic
    Fibrosis Disease Progression—Early Translation of a Predictive Algorithm
    into a Point-of-Care Tool. \textit{IEEE Journal of Translational Engineering
      in Health and Medicine.} Early Access. 2018.

	\item Rhonda Szczesniak, \textbf{Cole Brokamp}, Weiji Su, Gary L. McPhail, John Pestian, John P. Clancy. Early Detection of Rapid Cystic Fibrosis Disease Progression Tailored to Point of Care: A Proof-of-Principle Study. \textit{Healthcare Innovations and Point of Care Technologies}. (HI-POCT), 2017 IEEE. 204-207. 2017.

	\item Rhonda D. Szczesniak, Dan Li, Weiji Su, Cole Brokamp, John Pestian, Michael Seid, John P. Clancy. Phenotypes of Rapid Cystic Fibrosis Lung Disease Progression during Adolescence and Young Adulthood. American Journal of Respiratory And Critical Care Medicine. 196(4). 471-478. 2017.
	
\end{enumerate}

\end{enumerate}

\subsection*{Complete List of Published Work in MyBibliography:} 
\url{https://www.ncbi.nlm.nih.gov/myncbi/browse/collection/49821426}

\section{Research Support}

\subsection*{Ongoing Research Support}

\bigskip

\textbf{Internal Processes and Methods Award - Center for Clinical \&
Translational Science \& Training}\\
\emph{Using Machine Learning to Supplement Electronic Health Record
databases with Individual Socioeconomic Status}\\
Brokamp, PI (9/1/17 - 6/30/19)\\
Retrospective epidemiological studies are often created using electronic
health record databases. Although these records are ``wide'', they are
not ``deep'' with respect to individual level demographic data. We
propose a novel machine learning based approach that uses open city and
auditor databases to predict individual level income and family
socioeconomic status. This will solve the urgent problem of
unconfounding for individual SES in the execution of EHR based
research.\\
Role: PI

\bigskip

\textbf{NIH 5UG3OD023282-02}\\
\emph{Children's Respiratory Research and Environment Workgroup
(CREW)}\\
Gern, PI (9/01/2016 - 8/31/2023)\\
This consortium will identify asthma endotypes and overcome shortcomings
of individual cohorts by providing a large (nearly 9000 births and
long-term follow-up of 6000-7000 children and young adults) and diverse
national data set, harmonizing data related to asthma clinical
indicators and early life environmental exposures, developing
standardized measures for prospective data collection across CREW
cohorts and other ECHO studies, and conducting targeted enrollment of
additional subjects into existing cohorts.\\
Role: Co-I

\bigskip

\textbf{Ohio Department of Medicaid}\\
\emph{Ohio Opioid Analytics Project}\\
Hall, PI (5/14/18 - 5/30/19)\\
This project will develop and implement point-of-care predictive models
to identify risk factors for opioid endpoints in order to guide
clinicians and service delivery as well as identify interventions that
can be used to implement public health policies.\\
Role: Co-I

\bigskip

\textbf{NIH/NHLBI R01HL141286-01A1}\\
\emph{Mapping Environmental Contributions to Rapid Lung Disease
Progression in Cystic Fibrosis}\\
Sczcesniak, PI (1/1/19 - 12/31/23)\\
The overall objective of this research is to leverage a rich CF
registry, extant national and local environmental data sources, and
prospectively collected study data to accurately forecast the onset of
rapid decline progression.\\
Role: Co-I

\bigskip

\textbf{NIH/NIA R21AG057983}\\
\emph{A Novel Research Infrastructure Enabling Life-Course Studies of
Healthy Aging}\\
Woo/Urbina, PI (8/15/18 - 7/31/23)\\
The goal of this two-phase study is to develop the data and biospecimen
infrastructure for the Bogalusa Heart Study, the Princeton Lipid
Research Study and the NHLBI Growth and Health Study (R21 phase) and to
conduct pilot evaluations of the feasibility, acceptability and validity
of data collected using a variety of biometric sensors relating to
cardiometabolic risk, sleep quality and cognition in these cohorts (R33
phase). These two phases will together prepare these cohorts for future
aging-related studies.\\
Role: Co-I

\bigskip

\textbf{AHRQ PEDSnet K12}\\
\emph{Inpatient Screening for Parental Adversity and Resilience}\\
Shaw, PI (1/1/19 - 12/31/20)\\
This award will work to establish and implement a screening approach for
the assessment of parental adverse childhood experiences in the hospital
setting.\\
Role: Co-I

\bigskip

\textbf{NIH/NINDS R01 NS030678}\\
\emph{Comparison of Hemorrhagic \& Ischemic Stroke Among Blacks and
Whites}\\
Kleindorfer, PI (04/01/15 - 03/31/20)\\
Tracking of population-based stroke incidence in the Greater Cincinnati
and Northern Kentucky region, with special emphasis on stroke in the
young and stroke recurrence.\\
Role: Biostatistician

\bigskip

\subsection*{Recently Completed Research Support}

\bigskip

\textbf{Internal Arnold W. Strauss Fellowship Award - Cincinnati
Children's Hospital}\\
\emph{Assessing Exposure to Air Pollution Across Time and Space}\\
Brokamp, PI (7/1/16 - 6/30/17)\\
The primary objective of this award is to combine satellite-based
measurements, land use characteristics, and meteorologic data to create
a hybrid spatiotemporal model for ground level exposure to particulate
matter using exact addresses and dates.\\
Role: PI

\bigskip

\textbf{Internal Processes and Methods Award - Center for Clinical \&
Translational Science \& Training}\\
\emph{Validating a Geocoding Approach for Multi Site Studies}\\
Brokamp, PI (1/24/17 - 6/30/17)\\
The primary objective of this award is to compare the geocoding
(assigning latitude and longitude coordinates to addresses) accuracy of
our software DeGAUSS (DEcentralized Geomarker Assessment for mUlti Site
Studies) to with other common geocoding software. Furthermore, each
method will be evaluated based on it ability to correctly estimate
environmental exposures and community-level characteristics.\\
Role: PI

\bigskip

\end{document}
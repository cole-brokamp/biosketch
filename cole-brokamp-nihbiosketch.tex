%!TEX TS-program = xelatex
\documentclass{nihbiosketch}

%------------------------------------------------------------------------------

\name{Brokamp, Cole}
\eracommons{brokampr}
\position{Assistant Professor}

\begin{document}
%------------------------------------------------------------------------------

\begin{education}
University of Cincinnati  & B.S           & 06/2010  & Biomedical Engineering \\
University of Cincinnati               & Ph.D.         & 04/2016  & Biostatistics and Bioinformatics \\
Cincinnati Children's Hospital Medical Center  & Postdoctoral Research Fellow  & 10/2017  & Biostatistics and Epidemiology \\
\end{education}


\section{Personal Statement}

\begin{statement}

As a biostatistician, I have specialized myself in the areas of machine learning and its application to large environmental and clinical datasets.  The initial stages of this project is about reconstructing and assessing place-based characteristics for study participants in order to quantify their effects on cardiovascular, metabolic and neurologic dysfunction. Recent democratization of “big spatial data” and advances in geoinformatics have allowed unprecedented access to environmental characteristics that can be used to estimate exposures to pollutants that vary highly with respect to both time and space. More precise environmental features require more complex models and I have dedicated my early career to furthering exposure science methodology in order to bring more precise exposure assessments to environmental health studies. I have previously worked with Dr. Woo to geocode historical addresses and I also have experience in calculating location-based exposures and characteristics for epidemiological research cohorts. Furthermore, I have previously developed a novel approach and accompanying software package called DeGAUSS which allows for user-friendly attachment of geospatial variables to existing research cohorts. I will utilize this tool and my expertise to ensure that geocoding and assessment of environmental exposures and community characteristics is successfully achieved in a reproducible and accurate manner.\\

\begin{enumerate}
	
\item \textbf{Cole Brokamp}, Roman Jandarov, Monir Hossain, Patrick Ryan.
Predicting Daily Exposure to Urban Fine Particulate Matter at a High
Spatial Resolution. \emph{Under Review}.

\item \textbf{Cole Brokamp}, Chris Wolfe, Todd Lingren, John Harley, Patrick Ryan. Decentralized and Reproducible Geocoding and Characterization of Community and Environmental Exposures for Multi-Site Studies. \emph{Journal of American Medical Informatics Association.} \emph{In Press}.

\item \textbf{Cole Brokamp}, Roman Jandarov, MB Rao, Grace LeMasters, Patrick Ryan. Exposure assessment models for elemental components of particulate matter in an urban environment: A comparison of regression and random forest approaches. \emph{Atmospheric Environment}. 151. 1-11. 2017.
\end{enumerate}

\end{statement}

%------------------------------------------------------------------------------
\section{Positions and Honors}

\subsection*{Positions and Employment}
\begin{datetbl}
2010--2016  & Graduate Research Assistant, University of Cincinnati\\
2016--2017  & Research Fellow, Cincinnati Children's Hospital Medical Center\\
2017--      & Assistant Professor, Cincinnati Children's Hospital Medical Center\\
\end{datetbl}

\subsection*{Honors}
\begin{datetbl}
2010            & B.S. awarded with Distinguished Honors, University of Cincinnati\\
2016            & CCHMC Division of Biostatistics \& Epidemiology Travel Award\\
2016            & CCHMC Arnold W. Strauss Fellowship Award\\
2017            & CCHMC Epidemiology \& Biostatistics Top Publication \\
2017            & CCHMC Epidemiology \& Biostatistics Top Research Achievement \\
\end{datetbl}

%------------------------------------------------------------------------------

\section{Contribution to Science}

\begin{enumerate}


\item The main aim of my early career work has been to develop exposure assessment models for airborne pollutants based on machine learning techniques.  Specifically, applying random forest to land use models results in higher accuracy and precision of air pollution exposure assessment by elucidating complex interactions and nonlinear relationships between land use predictors and pollutant concentrations. This work includes the first machine learning or ensemble model used to assess exposure to elemental components of particulate matter. Recent introduction of remote sensing satellite data has allowed for extension of the land use random forest model to produce daily estimates of air pollution from 2000 - 2015 at a resolution of 1 x 1 km across the Greater Cincinnati area.

\begin{enumerate}
	
	\item \textbf{Cole Brokamp}, Roman Jandarov, MB Rao, Grace LeMasters, Patrick Ryan. Exposure assessment models for elemental components of particulate matter in an urban environment: A comparison of regression and random forest approaches. Atmospheric Environment. 151. 1-11. 2017.

	\item \textbf{Cole Brokamp}, Roman Jandarov, Monir Hossain, Patrick Ryan. Predicting Daily Exposure to Urban Fine Particulate Matter at a High Spatial Resolution. \emph{Under Review}.

\end{enumerate}

\item The work in developing LURF methods has helped me gain expertise in geospatial computing.  Along with other collaborators, I have applied these geospatial techniques to analyze the effect on health of other environmental exposures such as combined sewer overflows, elemental components of particulate matter, greenspace, and ozone.

\begin{enumerate}
	
	\item \textbf{Cole Brokamp}, Andrew F. Beck, Louis Muglia, Patrick Ryan. Combined Sewer Overflow Events and Childhood Emergency Department Visits: A Case-Crossover Study. \emph{Science of the Total Environment}. 607-608. 1180-1187. 2017.
	
	\item Lusine Yaghjyan, R Aroa, \textbf{Cole Brokamp}, E O'Meara, B Sprague, G Ghita, Patrick Ryan. Association of air pollution with mammographic breast density in the Breast Cancer Surveillance Consortium. \emph{Breast Cancer Research}. 19:36. 1-10. 2017.
	
	\item \textbf{Cole Brokamp}, MB Rao, Tina Zhihua Fan, Patrick H Ryan. Does the elemental composition of indoor and outdoor PM2.5 accurately represent the elemental composition of personal PM2.5?. \emph{Atmospheric Environment}. 101. 226-234. 2015.
	
	\item Rebecca Gernes, \textbf{Cole Brokamp}, Glenn Rice, J. Michael Wright,
	Michelle Kondo, Yvonne Michael, Geoffrey Donovan, Demetrios Gatziolis,
	David Bernstein, Grace LeMasters, James Lockey, G. Khurana Hershey,
	Patrick Ryan. Using medium- and high-resolution residential greenspace
	measures to assess risks of allergy outcomes in a cohort of children
	residing near Cincinnati, Ohio. \emph{Under Review}.
	
\end{enumerate}

\item Collaborating with other researchers wishing to use geospatial characteristics of research participants within existing cohorts and multisite studies has highlighted the significant need for a reproducible and distributed method for extracting place-based information from a residential address while maintaining the privacy of protected health information. I recently developed a novel approach and accompanying software package called DeGAUSS which overcomes the multiple challenges in the use of address data in multi-site studies and also serves as a more general reproducible research tool for geocoding and geomarker assessment. This approach is currently being implemented in a wide variety of national environmental health studies.

\begin{enumerate}

	\item \textbf{Cole Brokamp}, Chris Wolfe, Todd Lingren, John Harley, Patrick Ryan. Decentralized and Reproducible Geocoding and Characterization of Community and Environmental Exposures for Multi-Site Studies. \emph{Journal of American Medical Informatics Association.} \emph{In Press}.
	
	\item Rhonda D. Szczesniak, Dan Li, Weiji Su, \textbf{Cole Brokamp}, John Pestian, Michael Seid, John P. Clancy. Phenotypes of Rapid Cystic Fibrosis Lung Disease Progression during Adolescence and Young Adulthood. \emph{American Journal of Respiratory And Critical Care Medicine}. \emph{In Press}.
	
	\item Rhonda Szczesniak, \textbf{Cole Brokamp}, Weiji Su, Gary L. McPhail, John Pestian, John P. Clancy. Early Detection of Rapid Cystic Fibrosis Disease Progression Tailored to Point of Care: A Proof-of-Principle Study. \emph{Under Review}.

\end{enumerate}

\item As a trained biostatistician, I also have a general interest in collaborating with others and often use my expertise in spatial statistics to study the effect of place-based characteristics on health outcomes.

\begin{enumerate}

	\item Kelly J Brunst, Patrick H Ryan, \textbf{Cole Brokamp}, David Bernstein, Tiina Reponen, James Lockey, Gurjit K Khurana Hershey, Linda Levin, Sergey A Grinshpun, Grace LeMasters. Timing and duration of traffic-related air pollution exposure and the risk for childhood wheeze and asthma. American Journal of Respiratory and Critical Care Medicine. 192(4). 421-427. 2015.

	\item Jennifer Kannan, \textbf{Cole Brokamp}, David I. Bernstein, Grace K. LeMasters, Gurjit K. Khurana Hershey, Manuel Villareal, James E. Lockey, Patrick Ryan. Parental Snoring and Environmental Pollutants, but Not Aeroallergen Sensitization, Are Associated with Childhood Snoring in a Birth Cohort. Pediatric Allergy, Immunology, and Pulmonology. 0. 2016.
	
	\item Andrew F. Beck, Carley L. Riley, Stuart Taylor, \textbf{Cole Brokamp}, Robert S. Kahn. Toward a Culture of Health in Hospitals: Pervasive population disparities in inpatient bed-day rates across conditions and subspecialties. \emph{Under Review}.

\end{enumerate}

\end{enumerate}

\subsection*{Complete List of Published Work in MyBibliography:} 
\url{https://www.ncbi.nlm.nih.gov/myncbi/browse/collection/49821426}


%------------------------------------------------------------------------------

\section{Research Support}

\subsection*{Ongoing Research Support}

\bigskip

\textbf{Internal Processes and Methods Award - Center for Clinical \&
	Translational Science \& Training}\\
\emph{Using Machine Learning to Supplement Electronic Health Record
	databases with Individual Socioeconomic Status}\\
Brokamp, PI (9/1/17 - 12/31/17)\\
Retrospective epidemiological studies are often created using electronic
health record databases. Although these records are ``wide'', they are
not ``deep'' with respect to individual level demographic data. We
propose a novel machine learning based approach that uses open city and
auditor databases to predict individual level income and family
socioeconomic status. This will solve the urgent problem of
unconfounding for individual SES in the execution of EHR based
research.\\
Role: PI

\bigskip

\textbf{NIH/NIEHS 1R01ES019890-01}\\
\emph{Neurobehavioral and Neuroimaging Effects of Traffic Exposure in
	Children}\\
Ryan, PI (7/1/12 - 3/31/18)\\
The association between exposure to traffic-related air pollutants
(TRAP) during early childhood and neurobehavioral and neuroimaging
outcomes has not been thoroughly examined. The objective of the proposed
study is to determine if children exposed to increased levels of TRAP
during critical time periods of brain development have altered
neurobehavior in childhood as measured by a battery of valid and
reliable tests and to assess the physiologic impact of TRAP exposure on
brain structure, organization, and function using quantitative magnetic
resonance imaging (MRI). These results will fill important gaps in
current scientific knowledge related to the relationship between TRAP
exposure and neurobehavior and central nervous system effects.\\
Role: Biostatistician

\bigskip

\textbf{NIH 5K23AI121325}\\
\emph{Biomarkers and Risk Stratification in Pediatric Community}\\
Florin, PI (01/01/16 - 12/31/19)\\
The extensive variation in care, in addition to the lack of
evidence-based decision aids, highlights the critical need for an
improved understanding of disease severity and tools to guide management
for pediatric CAP. The proposed research will address this important
knowledge and practice gap.\\
Role: Biostatistician

\bigskip

\textbf{U01HG008666}\\
\emph{EMERGE: Better Outcomes for Children: Promoting Excellence in
	Healthcare Genomics to Inform Policy}\\
Harley, PI (09/01/15 - 05/31/19)\\
We have developed algorithms for the electronic health record (EHR), led
the Pediatric Workgroup, developed pharmacogenomics, evaluated the
preferences of parents and caregivers to advance genomic medicine and
assimilated technical advances into our EHR. The eMERGE effort has
become the basic fabric of the institutional initiative to incorporate
the extraordinary advances of genetics, genomics and the electronic
medical record into healthcare.\\
Role: Biostatistician

\bigskip

\textbf{Internal ARC - Cincinnati Children's Hospital}\\
\emph{Mother Infant Data Hub}\\
Marsolo, PI (7/1/15 - 7/1/18)\\
The goals of this award are to create a research database of
comprehensive clinical coverage for neonates born throughout the greater
Cincinnati area including linkage of medical records to external data
sets at the individual- and area-level during the first year of life.\\
Role: Biostatistician

\bigskip

\textbf{Internal ARC - Cincinnati Children's Hospital}\\
\emph{CARPE DIEM}\\
Ambroggio, PI (7/1/15 - 7/1/18)\\
The goals of this award are to develop a diagnostic tool based on the
urinary metabolome that can differentiate between viral and bacterial
community-acquired pneunomia in children.\\
Role: Biostatistician

\bigskip

\textbf{Internal - University of Cincinnati}\\
\emph{Epidemiology of Rural/Urban Disparities in Stroke}\\
Jasne, PI (1/1/17 - 12/31/17)\\
The goal of this project is to identify stroke incidence disparities
among rural and urban geographic areas.\\
Role: Biostatistician

%------------------------------------------------------------------------------

\subsection*{Recently Completed Research Support}

\bigskip

\textbf{Internal Arnold W. Strauss Fellowship Award - Cincinnati
	Children's Hospital}\\
\emph{Assessing Exposure to Air Pollution Across Time and Space}\\
Brokamp, PI (7/1/16 - 6/30/17)\\
The primary objective of this award is to combine satellite-based
measurements, land use characteristics, and meteorologic data to create
a hybrid spatiotemporal model for ground level exposure to particulate
matter using exact addresses and dates.\\
Role: PI

\bigskip

\textbf{Internal Processes and Methods Award - Center for Clinical \&
	Translational Science \& Training}\\
\emph{Validating a Geocoding Approach for Multi Site Studies}\\
Brokamp, PI (1/24/17 - 6/30/17)\\
The primary objective of this award is to compare the geocoding
(assigning latitude and longitude coordinates to addresses) accuracy of
our software DeGAUSS (DEcentralized Geomarker Assessment for mUlti Site
Studies) to with other common geocoding software. Furthermore, each
method will be evaluated based on it ability to correctly estimate
environmental exposures and community-level characteristics.\\
Role: PI

\bigskip

\textbf{Academy Health}\\
\emph{Community Health Peer Learning Program: Participant Community}\\
Beck, PI (2/1/16 - 6/30/17)\\
The goals of this project are to reduce by 10\% the inpatient bed-day
rate for one high risk neighborhood in Cincinnati through interventions
promoted by shared data and improved data visualization.\\
Role: Biostatistician

\bigskip

\textbf{HEI 4784-RFA08-1/09-5}\\
\emph{Analysis of Personal and Home Characteristics Associated with the
	Elemental Composition of PM2.5 in Indoor, Outdoor, and Personal Air in
	the RIOPA Study}\\
Ryan, PI (12/1/12 - 11/30/13)\\
The purpose of this study is to assess the relationship between
concurrent measurements of the elemental composition of PM2.5 in indoor,
outdoor, and ambient air and the elemental composition of indoor,
outdoor and personal air across individuals and cities. The study will
also identify personal, home, and environmental factors significantly
associated with specific elements or clusters of elements in PM2.5.\\
Role: Biostatistician

\bigskip

\textbf{Gerber Pediatric Research Grant, Gerber Foundation}\\
\emph{Clinical Prediction Model for Community-Acquired Pneumonia}\\
Florin, PI (1/1/14 - 12/31/16)\\
This project will use clinical data and the biomarker procalcitonin to
develop a severity score used to predict the development of severe
disease and complications in children with community-acquired pneumonia,
the most common serious bacterial infection children and leading killer
of children worldwide.\\
Role: Biostatistician



\end{document}

%!TEX TS-program = xelatex
\documentclass{nihbiosketch}

%------------------------------------------------------------------------------

\name{Brokamp, Cole}
\eracommons{brokampr}
\position{Assistant Professor}

\begin{document}
%------------------------------------------------------------------------------

\begin{education}
University of Cincinnati  & B.S           & 06/2010  & Biomedical Engineering \\
University of Cincinnati               & Ph.D.         & 04/2016  & Biostatistics and Bioinformatics \\
Cincinnati Children's Hospital Medical Center  & Postdoctoral Research Fellow  & 10/2017  & Biostatistics and Epidemiology \\
\end{education}

\section{Personal Statement}

\begin{statement}

  As a biostatistician and geoinformatician, I have specialized myself in the areas of machine learning and its application to large environmental and clinical datasets. Recent democratization of “big spatial data” and advances in geoinformatics have allowed unprecedented access to environmental and socioeconomic characteristics that vary highly with respect to both time and space. More precise environmental features require more complex modeling and I have dedicated my early career to furthering exposure science methodology in order to bring more precise exposure assessment tools to environmental health studies. Furthermore, I have developed a novel approach and accompanying software package called DeGAUSS which allows for user-friendly attachment of geospatial variables to existing research cohorts while mitigating key privacy challenges. I am the founding director of the Geospatial Research Accelerator for Precision Population Health (GRAPPH), which is a shared facility at Cincinnati Children's Hospital Medical Center that works to develop and democratize geospatial data and methodologies across the institution.

I have also developed and maintain a ``deprivation index'', which is a census-tract level indicator of neighborhood material deprivation. It utilizes American Community Survey data related to income, education, poverty, and health insurance. This index has been previously validated with respect to pediatric health outcomes and has been applied in several different multi-site studies.


\begin{enumerate}

\item \textbf{Cole Brokamp}, Chris Wolfe, Todd Lingren, John Harley, Patrick Ryan. Decentralized and Reproducible Geocoding and Characterization of Community and Environmental Exposures for Multi-Site Studies. \textit{Journal of American Medical Informatics Association.} 25(3). 309-314. 2018.
  
\item \textbf{Cole Brokamp}. DeGAUSS: Decentralized Geomarker Assessment for Multi-Site Studies. \textit{Journal of Open Source Software}. 2018. 

\item \textbf{Cole Brokamp}, Andrew F. Beck, Neera K. Goyal, Patrick Ryan,
  James M. Greenberg, Eric S. Hall. Material Community Deprivation and
  Hospital Utilization During the First Year of Life: An Urban
  Population-Based Cohort Study. \textit{Annals of Epidemiology}. 30. 37-43.
  2019.

\end{enumerate}

\end{statement}

%------------------------------------------------------------------------------
\section{Positions, Scientific Appointments, and Honors}

\subsection*{Positions and Scientific Appointments}
\begin{datetbl}
2012--2016 & Research Associate, Department of Environmental Health, University of Cincinnati \\	
2016--2017  & Research Fellow, Cincinnati Children's Hospital Medical Center Division of Biostatistics \& Epidemiology\\
2017--      & Assistant Professor of Pediatrics, the University of Cincinnati Department of Pediatrics and Cincinnati Children’s Hospital Medical Center Division of Biostatistics \& Epidemiology\\
\end{datetbl}

\subsection*{Honors}
\begin{datetbl}
2010            & B.S. awarded with Distinguished Honors, University of Cincinnati\\
2016            & CCHMC Division of Biostatistics \& Epidemiology Travel Award\\
2016            & CCHMC Arnold W. Strauss Fellowship Award\\
2017            & CCHMC Epidemiology \& Biostatistics Top Publication and Top
Research Achievement\\
2020            & CCHMC Epidemiology \& Biostatistics Top Publication \\
\end{datetbl}

%------------------------------------------------------------------------------

\section{Contributions to Science}

\begin{enumerate}

\item My early career has been spent developing spatiotemporal exposure assessment models for
  enivronmental pollutants and community characteristics based on machine learning
  techniques.  This work includes the first machine
  learning or ensemble model used to assess exposure to elemental components of
  particulate matter. Recent introduction of remote sensing satellite data has
  allowed for extension of the land use random forest model to produce daily
  estimates of air pollution back to 2000 at a resolution of 1 x 1 km.

\begin{enumerate}
	
  \item \textbf{Cole Brokamp}, Eric B. Brandt, Patrick H. Ryan. Assessing
  Exposure to Outdoor Air Pollution for Epidemiological Studies:
  Model-based and Personal Sampling Strategies. \emph{Journal of Allergy
    and Clinical Immunology}. 2019.

	\item \textbf{Cole Brokamp}, Roman Jandarov, Monir Hossain, Patrick Ryan. Predicting Daily Urban Fine Particulate Matter Concentrations Using Random Forest. \textit{Environmental Science \& Technology}. 52 (7). 4173-4179. 2018.

	\item \textbf{Cole Brokamp}, Roman Jandarov, MB Rao, Grace LeMasters, Patrick Ryan. Exposure assessment models for elemental components of particulate matter in an urban environment: A comparison of regression and random forest approaches. \textit{Atmospheric Environment}. 151. 1-11. 2017.
	
	\item \textbf{Cole Brokamp}, MB Rao, Patrick Ryan, Roman Jandarov. A comparison of resampling and recursive partitioning methods in random forest for estimating the asymptotic variance using the infinitesimal jackknife. \textit{Stat}. 6(1). 360-372. 2017.

  \item \textbf{Cole Brokamp}, Grace LeMasters, Patrick Ryan. Residential
    mobility impacts exposure assessment and community socioeconomic
    characteristics in longitudinal epidemiology studies. \emph{Journal of
      Exposure Science and Environmental Epidemiology}. 26(4). 428-34. 2016.

  \item \textbf{Cole Brokamp}, MB Rao, Tina Zhihua Fan, Patrick H Ryan. Does the
    elemental composition of indoor and outdoor PM2.5 accurately represent
    the elemental composition of personal PM2.5?. \emph{Atmospheric
      Environment}. 101. 226-234. 2015.

\end{enumerate}

\item Building on advanced exposure assessment models has allowed me to lead epidemiological studies on the impacts of the built environment (e.g., fine particulate matter, greenspace, combined sewer overflows, community deprivation) on several different pediatric health outcomes (e.g., psychiatric, neurobehavioral, gastrointestinal, and all-cause hospital utilization).

  \begin{enumerate}

  \item \textbf{Cole Brokamp}, Jeffrey R. Strawn, Andrew F. Beck, Pat Ryan.
    Pediatric Psychiatric Emergency Department Utilization and Fine
    Particulate Matter: A Case-Crossover Study. \textit{Environmental Health
      Perspectives}. 2019.

  \item Juliana Madzia, Patrick Ryan, Kimberly Yolton, Zana Percy, Nick Newman, Grace
    LeMasters, \textbf{Cole Brokamp}. Residential Greenspace Is Associated with Childhood
    Behavioral Outcomes. \textit{Journal of Pediatrics}. 30. 37-43. 2019.

  \item \textbf{Cole Brokamp}, Andrew F. Beck, Neera K. Goyal, Patrick Ryan,
    James M. Greenberg, Eric S. Hall. Material Community Deprivation and
    Hospital Utilization During the First Year of Life: An Urban
    Population-Based Cohort Study. \textit{Annals of Epidemiology}. 30. 37-43.
    2019.

  \item \textbf{Cole Brokamp}, Andrew F. Beck, Louis Muglia, Patrick Ryan. Combined Sewer Overflow Events and Childhood Emergency Department Visits: A Case-Crossover Study. \textit{Science of the Total Environment}. 607-608. 1180-1187. 2017.
    
  \end{enumerate}

\item I have developed a novel approach and accompanying software package called DeGAUSS
  which overcomes multiple privacy-related challenges in the use of address data in
  multi-site studies and also serves as a more general reproducible and scalable
  research tool for geocoding and geomarker assessment. This approach is currently being
  implemented in a wide variety of national environmental health studies. Extending this
  approach into a scalable and sustainable framework for automated integration of
  disparate and heterogeneous geomarkers via spatiotemporal location has reduced
  the need for manual data curation and specialized expertise required
  to utilize them within biomedical research studies.

\begin{enumerate}
	
	\item \textbf{Cole Brokamp}, Chris Wolfe, Todd Lingren, John Harley, Patrick Ryan. Decentralized and Reproducible Geocoding and Characterization of Community and Environmental Exposures for Multi-Site Studies. \textit{Journal of American Medical Informatics Association.} 25(3). 309-314. 2018.
	
	\item \textbf{Cole Brokamp}. DeGAUSS: Decentralized Geomarker Assessment for Multi-Site Studies. \textit{Journal of Open Source Software}. 2018. 

\end{enumerate}

\item I have also contributed to several studies on the disparities of health outcomes within children and the contribution of the place-based and social determinants of health to these disparities in order to identify root causes and meaningful solutions.

\begin{enumerate}

  \item Erica Andrist, \textbf{Cole Brokamp}, Stuart Taylor, Carley Riley,
  Andrew Beck. Neighborhood Poverty and Pediatric Intensive Care Use.
  \textit{Pediatrics}. 2019.

	\item Andrew F. Beck, Carley L. Riley, Stuart Taylor, \textbf{Cole Brokamp},
    Robert S. Kahn. Toward a Culture of Health in Hospitals: Pervasive
    population disparities in inpatient bed-day rates across conditions and
    subspecialties. \textit{Health Affairs}. 37(4). 551-559. 2018.
		
	\item Lauren C. Riney, \textbf{Cole Brokamp}, Andrew F. Beck, Wendy Pomerantz,
    Hamilton Schwartz, Todd A. Florin. Emergency Medical Services Utilization is
    Associated with Community Deprivation in Children. \textit{Prehospital
      Emergency Care}. 2018.

\end{enumerate}

\item Lastly, I have contributed to a research team that has recently used
  functional data analysis combined with joint modeling (FD-JM) to identify and
  predict rapid decline in lung function among patients with cystic fibrosis
  (CF) lung disease. My work in translating this predictive model into an interactive application has allowed for patients and clinicians to take advantage of it at the bedside.  Focus groups and partnerships with clinicians have allowed us to iteratively develop the application based on end-user feedback. Work with the CF Foundation Patient Registry (CFFPR) to implement these models and visualizations into clinical settings has improved prognostic care.

\begin{enumerate}

  \item Christopher Wolfe, Teresa Pestian, Emrah Gecili, Weiji Su, Ruth H. Keogh, John P. Pestian, Michael Seid, Peter J. Diggle, Assem Ziady, John P. Clancy, Daniel H. Grossoehme, Rhonda D. Szczesniak, \textbf{Cole Brokamp}. Cystic Fibrosis Point of Personalized Detection (CFPOPD): An Interactive Web Application. \emph{JMIR Med Inform}. 8(12):e23530. 2020.

  \item Rhonda D. Szczesniak, Weiji Su, \textbf{Cole Brokamp}, Ruth H. Keogh,
  John P. Pestian, Michael Seid, Peter J. Diggle, John P. Clancy. Dynamic
  predictive probabilities to monitor rapid cystic fibrosis disease
  progression. \emph{Statistics in Medicine}. 2019.

	\item Rhonda D. Szczesniak, \textbf{Cole Brokamp}, Weiji Su, Gary L. McPhail,
    John Pestian, and John P. Clancy. Improving Detection of Rapid Cystic
    Fibrosis Disease Progression—Early Translation of a Predictive Algorithm
    into a Point-of-Care Tool. \textit{IEEE Journal of Translational Engineering
      in Health and Medicine.} 7(1). 1-8. 2019.

	\item Rhonda Szczesniak, \textbf{Cole Brokamp}, Weiji Su, Gary L. McPhail, John Pestian, John P. Clancy. Early Detection of Rapid Cystic Fibrosis Disease Progression Tailored to Point of Care: A Proof-of-Principle Study. \textit{Healthcare Innovations and Point of Care Technologies}. (HI-POCT), 2017 IEEE. 204-207. 2017.

	\item Rhonda D. Szczesniak, Dan Li, Weiji Su, \textbf{Cole Brokamp}, John Pestian, Michael Seid, John P. Clancy. Phenotypes of Rapid Cystic Fibrosis Lung Disease Progression during Adolescence and Young Adulthood. \textit{American Journal of Respiratory And Critical Care Medicine}. 196(4). 471-478. 2017.

\item US Patent: Assem Ziady, Rhonda Szczesniak, John Clancy, \textbf{Cole Brokamp}, inventors; Cincinnati Children's Hospital Medical Center, assignee. Compositions and methods for treatment of lung function. United States patent US 10,761,099. 2020 Sep 1.

  \item Software Application: Cystic Fibrosis Point of Personalized Detection (CFPOPD) for Forecasting Rapid Decline: http://predictfev1.com Co-Developers: Clancy, Szczesniak.
  
\end{enumerate}

\end{enumerate}

\subsection*{Complete List of Published Work in ORCiD:} 
\url{https://orcid.org/0000-0002-0289-3151}

\section{Additional Information: Research Support and/or Scholastic Performance}

\subsection*{Ongoing Research Support}

\bigskip

\textbf{NIH/NLM 1R01LM013222-01A1}\\
\emph{A Framework for Automated and Reproducible Geomarker Curation and Computation at Scale}\\
Brokamp, PI (08/01/2020 – 07/31/2024)\\
This award will create a framework for developing a standardized, free and open source library of reproducible and computable geomarkers that will enhance the efficiency and collaboration of biomedical researchers utilizing place-based data at scale.\\
Role: PI

\bigskip

\textbf{ECHO Opportunities and Infrastructure Fund Award}\\
\emph{Decentralized and Reproducible Geomarker Assessment for Multi-Site
  Studies}\\
Brokamp, PI (09/01/2019 - 08/31/2021)\\
This award will work towards building geospatial exposure assessment
computing tools for utilizing high resolution spatiotemporal gridded
datasets within ECHO.\\
Role: PI

\bigskip

\textbf{NIH/NHLBI R01HL141286-01A1}\\
\emph{Mapping Environmental Contributions to Rapid Lung Disease
Progression in Cystic Fibrosis}\\
Sczcesniak, PI (01/01/2019 - 12/31/2023)\\
The overall objective of this research is to leverage a rich CF
registry, extant national and local environmental data sources, and
prospectively collected study data to accurately forecast the onset of
rapid decline progression.\\
Role: Co-I

\bigskip

\textbf{NIH/NHGRI U01HG011172}\\
\emph{Polygenic Risk Scores for Healthier African American Families}\\
Harley, PI (04/01/2020 - 3/31/2025)\\
We will ascertain and enroll 800 African American mothers with newborn babies along with available fathers and siblings and develop polygenic risk scores and incorporate them into genomic risk estimates for Asthma, Atopic Dermatitis, Obesity, Hypertension, Hypercholesterolemia, Premature Birth, and Breast Cancer. We will cope with the ethics of returning results and for selected situations intervene for mitigate risk.\\
Role: Co-I

\bigskip

\textbf{NIH/NIEHS 1R01ES031054-01A1}\\
\emph{Epigenetics, Air Pollution, and Childhood Mental Health}\\
Brunst, PI (07/01/2020 - 04/31/2025)\\
This project utilizes data from three longitudinal birth cohorts to examine the impact of air pollution on the epigenome and the onset of childhood anxiety and depression symptoms. DNA methylation biomarkers are investigated to advance our understanding of potential molecular pathways involved in air pollution neurotoxicity and/or anxiety and depression pathophysiology.\\
Role: Co-I

\bigskip

\subsection*{Recently Completed Research Support}

\bigskip

\textbf{ODH Contract No. CSP907820}\\
\emph{Model Identifying Geographic Areas in Ohio for Blood Lead Testing}\\
Brokamp, PI (04/01/2020 - 09/30/2020)\\
This award will develop a predictive model to determine which children should be
tested for potentially high blood lead during physician visits based on their
residential location. \\
Role: PI

\bigskip

\textbf{Internal Processes and Methods Award - Center for Clinical \&
  Translational Science \& Training}\\
\emph{Using Machine Learning to Supplement Electronic Health Record
  databases with Individual Socioeconomic Status}\\
Brokamp, PI (09/1/2017 - 06/30/2019)\\
Retrospective epidemiological studies are often created using electronic
health record databases that lack individual level demographic data. We
propose a novel machine learning based approach that uses open city and
auditor databases to predict individual level income and family
socioeconomic status.\\
Role: PI

\bigskip

\textbf{Internal Arnold W. Strauss Fellowship Award - Cincinnati
Children's Hospital}\\
\emph{Assessing Exposure to Air Pollution Across Time and Space}\\
Brokamp, PI (07/01/16 - 06/30/17)\\
The primary objective of this award is to combine satellite-based
measurements, land use characteristics, and meteorologic data to create
a hybrid spatiotemporal model for ground level exposure to particulate
matter using exact addresses and dates.\\
Role: PI

\bigskip

\textbf{Internal Processes and Methods Award - Center for Clinical \&
Translational Science \& Training}\\
\emph{Validating a Geocoding Approach for Multi Site Studies}\\
Brokamp, PI (01/24/17 - 06/30/17)\\
The primary objective of this award is to compare the geocoding
(assigning latitude and longitude coordinates to addresses) accuracy of
our software DeGAUSS (DEcentralized Geomarker Assessment for mUlti Site
Studies) to with other common geocoding software. Furthermore, each
method will be evaluated based on it ability to correctly estimate
environmental exposures and community-level characteristics.\\
Role: PI

\bigskip

\end{document}

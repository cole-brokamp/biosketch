%!TEX TS-program = xelatex
\documentclass{nihbiosketch}

%------------------------------------------------------------------------------

\name{Brokamp, Cole}
\eracommons{brokampr}
\position{Assistant Professor}

\begin{document}
%------------------------------------------------------------------------------

\begin{education}
University of Cincinnati  & B.S           & 06/2010  & Biomedical Engineering \\
University of Cincinnati               & Ph.D.         & 04/2016  & Biostatistics and Bioinformatics \\
Cincinnati Children's Hospital Medical Center  & Postdoctoral Research Fellow  & 10/2017  & Biostatistics and Epidemiology \\
\end{education}

\section{Personal Statement}

\begin{statement}

I am an environmental and population health scientist trained at the
intersection of biostatistics, epidemiology, and geoinformatics in order to
advance precision public health. My work has focused on using
machine learning and causal inference methods with ``big spatial data'' to
estimate the effect of environmental exposures (e.g., air pollution, combined
sewer overflows, greenspace, poverty, cultural isolation) on pediatric health
outcomes (e.g., all-cause hospital and emergency medical services utilization,
psychiatric disorder exacerbations, neurobehavioral problems, asthma
exacerbations and development). Specifically, my interest lies in the
interaction of these urban exposures with individual- and community-level
socioeconomic characteristics in order to to identify subpopulations of children
who are more susceptible to environmental health problems and inform targeted
interventions. I have specialized myself in the areas of machine learning and geoinformatics and their applications to large environmental and
clinical datasets. My early career work was dedicated to furthering exposure science methodology in
order to bring more precise exposure assessment tools to environmental health
studies. Furthermore, I have developed a novel approach and
accompanying software package called DeGAUSS which allows for user-friendly
attachment of geospatial variables to existing research cohorts and within
multi-site studies where sharing of private health information like residential
addresses is not feasible. 

I have created a satellite-based land use random forest model
that produces daily estimates of fine particulate matter from 2000 through 2015 at a
resolution of 1 x 1 km across the Greater Cincinnati area.  I will continue to
utilize this model to assign PM$_{2.5}$ exposures to study
participants in collaboration with Dr. Brunst as well as work to extend the
models to more recent years. I have previously collaborated with Drs. Ryan and Brunst
within the CCAAPS cohort, studying the health effects of traffic related air
pollution.  Lastly, I also have experience with distributed lag models, specifically studying how
the effects of temporal environmental exposures such as combined sewer overflows
and fine particulate matter impact the risk of a health outcome at different
lagged times and cumulatively.

In this project, I will serve as co-investiagor and will utilize my training and expertise in statistical modeling
with respect to lagged environmental exposures and health outcomes to supervise
and work alongside a postdoctoral fellow who will be responsible for (1) updating and creating
exposure assessment models, (2) conducting temporal exposure assessment within the
CCAAPS cohort, (3) analyzing the lagged effects of pollutant exposures on
epigenetic and psychosocial outcomes, and (4) help to develop and implement novel
methods for controlling the false discovery rate when using distributed lag
models with epigenome-wide data.

In summary, my experience and expertise with the CCAAPS cohort, exposure assessment and
modeling, as well as statistical methods for evaluating the lagged effects of
environmental exposures on health outcomes will benefit the proposed project. I
look forward to contributing to Dr. Brunst's important, timely, and impactful
research proposal.\\

  \begin{enumerate}

  \item \textbf{Cole Brokamp}, Andrew F. Beck, Louis Muglia, Patrick Ryan. Combined Sewer Overflow Events and Childhood Emergency Department Visits: A Case-Crossover Study. \textit{Science of the Total Environment}. 607-608. 1180-1187. 2017.

  \item \textbf{Cole Brokamp}, Jeffrey R. Strawn, Andrew F. Beck, Pat Ryan.
    Pediatric Psychiatric Emergency Department Utilization and Fine
    Particulate Matter: A Case-Crossover Study. \textit{Environmental Health
      Perspectives}. 2019.

	\item \textbf{Cole Brokamp}, Roman Jandarov, Monir Hossain, Patrick Ryan. Predicting Daily Urban Fine Particulate Matter Concentrations Using Random Forest. Environmental Science \& Technology. 52 (7). 4173-4179. 2018.

  \item \textbf{Cole Brokamp}, Eric B. Brandt, Patrick H. Ryan. Assessing
    Exposure to Outdoor Air Pollution for Epidemiological Studies:
    Model-based and Personal Sampling Strategies. \emph{Journal of Allergy
      and Clinical Immunology}. 2019.

  \end{enumerate}

\end{statement}

%------------------------------------------------------------------------------
\section{Positions and Honors}

\subsection*{Positions and Employment}
\begin{datetbl}
2012--2016 & Research Associate, Department of Environmental Health, University of Cincinnati \\	
2016--2017  & Research Fellow, Cincinnati Children's Hospital Medical Center Division of Biostatistics \& Epidemiology\\
2017--      & Assistant Professor of Pediatrics, the University of Cincinnati Department of Pediatrics and Cincinnati Children’s Hospital Medical Center Division of Biostatistics \& Epidemiology\\
\end{datetbl}

\subsection*{Honors}
\begin{datetbl}
2010            & B.S. awarded with Distinguished Honors, University of Cincinnati\\
2016            & CCHMC Division of Biostatistics \& Epidemiology Travel Award\\
2016            & CCHMC Arnold W. Strauss Fellowship Award\\
2017            & CCHMC Epidemiology \& Biostatistics Top Publication \\
2017            & CCHMC Epidemiology \& Biostatistics Top Research Achievement \\
\end{datetbl}

%------------------------------------------------------------------------------

\section{Contribution to Science}

\begin{enumerate}

\item My early career was spent developing exposure assessment models for
  spatial pollutants and community characteristics based on machine learning
  techniques.  Specifically, applying random forest to land use models results
  in higher accuracy and precision of air pollution exposure assessment by
  elucidating complex interactions and nonlinear relationships between land use
  predictors and pollutant concentrations. This work includes the first machine
  learning or ensemble model used to assess exposure to elemental components of
  particulate matter. Recent introduction of remote sensing satellite data has
  allowed for extension of the land use random forest model to produce daily
  estimates of air pollution back to 2000 at a resolution of 1 x 1 km.

\begin{enumerate}
	
  \item \textbf{Cole Brokamp}, Eric B. Brandt, Patrick H. Ryan. Assessing
  Exposure to Outdoor Air Pollution for Epidemiological Studies:
  Model-based and Personal Sampling Strategies. \emph{Journal of Allergy
    and Clinical Immunology}. 2019.

	\item \textbf{Cole Brokamp}, Roman Jandarov, Monir Hossain, Patrick Ryan. Predicting Daily Urban Fine Particulate Matter Concentrations Using Random Forest. Environmental Science \& Technology. 52 (7). 4173-4179. 2018.

	\item \textbf{Cole Brokamp}, Roman Jandarov, MB Rao, Grace LeMasters, Patrick Ryan. Exposure assessment models for elemental components of particulate matter in an urban environment: A comparison of regression and random forest approaches. \textit{Atmospheric Environment}. 151. 1-11. 2017.
	
	\item \textbf{Cole Brokamp}, MB Rao, Patrick Ryan, Roman Jandarov. A comparison of resampling and recursive partitioning methods in random forest for estimating the asymptotic variance using the infinitesimal jackknife. \textit{Stat}. 6(1). 360-372. 2017.

\end{enumerate}

\item Building on advanced exposure assessment models has allowed me to lead epidemiological studies on the impacts of the built environment (e.g., fine particulate matter, greenspace, combined sewer overflows, community deprivation) on several different pediatric health outcomes (e.g., psychiatric, neurobehavioral, gastrointestinal, and all-cause hospital utilization).

  \begin{enumerate}

  \item \textbf{Cole Brokamp}, Jeffrey R. Strawn, Andrew F. Beck, Pat Ryan.
    Pediatric Psychiatric Emergency Department Utilization and Fine
    Particulate Matter: A Case-Crossover Study. \textit{Environmental Health
      Perspectives}. 2019.

  \item Juliana Madzia, Patrick Ryan, Kimberly Yolton, Zana Percy, Nick Newman, Grace
    LeMasters, \textbf{Cole Brokamp}. Residential Greenspace Is Associated with Childhood
    Behavioral Outcomes. \textit{Journal of Pediatrics}. 30. 37-43. 2019.

  \item \textbf{Cole Brokamp}, Andrew F. Beck, Neera K. Goyal, Patrick Ryan,
    James M. Greenberg, Eric S. Hall. Material Community Deprivation and
    Hospital Utilization During the First Year of Life: An Urban
    Population-Based Cohort Study. \textit{Annals of Epidemiology}. 30. 37-43.
    2019.

  \item \textbf{Cole Brokamp}, Andrew F. Beck, Louis Muglia, Patrick Ryan. Combined Sewer Overflow Events and Childhood Emergency Department Visits: A Case-Crossover Study. \textit{Science of the Total Environment}. 607-608. 1180-1187. 2017.
    
  \end{enumerate}

\item Collaborating with other researchers wishing to use geospatial
  characteristics of research participants within existing cohorts and
  multi-site studies has highlighted the significant need for a reproducible and
  distributed method for extracting place-based information from a residential
  address while maintaining the privacy of protected health information. I have
  developed a novel approach and accompanying software package called DeGAUSS
  which overcomes the multiple challenges in the use of address data in
  multi-site studies and also serves as a more general reproducible research
  tool for geocoding and geomarker assessment. This approach is currently being
  implemented in a wide variety of national environmental health studies. Extending this
  approach into a scalable and sustainable framework for automated integration of
  disparate and heterogeneous geomarkers via spatiotemporal location has reduced
  the need for manual data curation and specialized expertise required
  to utilize them within biomedical research studies.

\begin{enumerate}
	
	\item \textbf{Cole Brokamp}, Chris Wolfe, Todd Lingren, John Harley, Patrick Ryan. Decentralized and Reproducible Geocoding and Characterization of Community and Environmental Exposures for Multi-Site Studies. \textit{Journal of American Medical Informatics Association.} 25(3). 309-314. 2018.
	
	\item \textbf{Cole Brokamp}. DeGAUSS: Decentralized Geomarker Assessment for Multi-Site Studies. \textit{Journal of Open Source Software}. 2018. 
	
\end{enumerate}

\item I have also contributed to several studies on the disparities of health outcomes within children and the contribution of the place-based and social determinants of health to these disparities in order to identify root causes and meaningful solutions.

\begin{enumerate}

  \item Erica Andrist, \textbf{Cole Brokamp}, Stuart Taylor, Carley Riley,
  Andrew Beck. Neighborhood Poverty and Pediatric Intensive Care Use.
  \textit{Pediatrics}. 2019.

	
	\item Andrew F. Beck, Carley L. Riley, Stuart Taylor, \textbf{Cole Brokamp},
    Robert S. Kahn. Toward a Culture of Health in Hospitals: Pervasive
    population disparities in inpatient bed-day rates across conditions and
    subspecialties. \textit{Health Affairs}. 37(4). 551-559. 2018.
		
	\item Lauren C. Riney, \textbf{Cole Brokamp}, Andrew F. Beck, Wendy Pomerantz,
    Hamilton Schwartz, Todd A. Florin. Emergency Medical Services Utilization is
    Associated with Community Deprivation in Children. \textit{Prehospital
      Emergency Care}. 2018.

\end{enumerate}

\item Lastly, I have contributed to a research team that has recently used
  functional data analysis combined with joint modeling (FD-JM) to identify and
  predict rapid decline in lung function among patients with cystic fibrosis
  (CF) lung disease. My work in translating this predictive model into an interactive application has allowed for patients and clinicians to take advantage of it at the bedside.  Focus groups and partnerships with clinicians have allowed us to iteratively develop the application based on end-user feedback. Work with the CF Foundation Patient Registry (CFFPR) to implement these models and visualizations into clinical settings has improved prognostic care.

\begin{enumerate}
	
	\item Rhonda D. Szczesniak, \textbf{Cole Brokamp}, Weiji Su, Gary L. McPhail,
    John Pestian, and John P. Clancy. Improving Detection of Rapid Cystic
    Fibrosis Disease Progression—Early Translation of a Predictive Algorithm
    into a Point-of-Care Tool. \textit{IEEE Journal of Translational Engineering
      in Health and Medicine.} 7(1). 1-8. 2019.

	\item Rhonda Szczesniak, \textbf{Cole Brokamp}, Weiji Su, Gary L. McPhail, John Pestian, John P. Clancy. Early Detection of Rapid Cystic Fibrosis Disease Progression Tailored to Point of Care: A Proof-of-Principle Study. \textit{Healthcare Innovations and Point of Care Technologies}. (HI-POCT), 2017 IEEE. 204-207. 2017.

	\item Rhonda D. Szczesniak, Dan Li, Weiji Su, \textbf{Cole Brokamp}, John Pestian, Michael Seid, John P. Clancy. Phenotypes of Rapid Cystic Fibrosis Lung Disease Progression during Adolescence and Young Adulthood. \textit{American Journal of Respiratory And Critical Care Medicine}. 196(4). 471-478. 2017.
	
\end{enumerate}

\end{enumerate}

\subsection*{Complete List of Published Work in ORCiD:} 
\url{https://orcid.org/0000-0002-0289-3151}

\section{Research Support}

\subsection*{Ongoing Research Support}

\bigskip

\textbf{ECHO Opportunities and Infrastructure Fund Award}\\
\emph{Decentralized and Reproducible Geomarker Assessment for Multi-Site
  Studies}\\
Brokamp, PI (09/01/2019 - 08/31/2021)\\
This award will work towards building geospatial exposure assessment
computing tools for utilizing high resolution spatiotemporal gridded
datasets within ECHO.\\
Role: PI

\bigskip

\textbf{NIH 5UG3OD023282-02}\\
\emph{Children's Respiratory Research and Environment Workgroup
(CREW)}\\
Gern, PI (9/01/2016 - 8/31/2023)\\
This consortium will identify asthma endotypes and overcome shortcomings
of individual cohorts by providing a large (nearly 9000 births and
long-term follow-up of 6000-7000 children and young adults) and diverse
national data set, harmonizing data related to asthma clinical
indicators and early life environmental exposures, developing
standardized measures for prospective data collection across CREW
cohorts and other ECHO studies, and conducting targeted enrollment of
additional subjects into existing cohorts.\\
Role: Co-I

\bigskip

\textbf{NIH/NHLBI R01HL141286-01A1}\\
\emph{Mapping Environmental Contributions to Rapid Lung Disease
Progression in Cystic Fibrosis}\\
Sczcesniak, PI (1/1/19 - 12/31/23)\\
The overall objective of this research is to leverage a rich CF
registry, extant national and local environmental data sources, and
prospectively collected study data to accurately forecast the onset of
rapid decline progression.\\
Role: Co-I

\bigskip

\textbf{NIH/NIA R21AG057983}\\
\emph{A Novel Research Infrastructure Enabling Life-Course Studies of
Healthy Aging}\\
Woo/Urbina, PI (8/15/18 - 7/31/23)\\
The goal of this two-phase study is to develop the data and biospecimen
infrastructure for the Bogalusa Heart Study, the Princeton Lipid
Research Study and the NHLBI Growth and Health Study (R21 phase) and to
conduct pilot evaluations of the feasibility, acceptability and validity
of data collected using a variety of biometric sensors relating to
cardiometabolic risk, sleep quality and cognition in these cohorts (R33
phase). These two phases will together prepare these cohorts for future
aging-related studies.\\
Role: Co-I

\bigskip

\textbf{AHRQ PEDSnet K12}\\
\emph{Inpatient Screening for Parental Adversity and Resilience}\\
Shaw, PI (1/1/19 - 12/31/20)\\
This award will work to establish and implement a screening approach for
the assessment of parental adverse childhood experiences in the hospital
setting.\\
Role: Co-I

\bigskip

\textbf{NIH/NINDS R01 NS030678}\\
\emph{Comparison of Hemorrhagic \& Ischemic Stroke Among Blacks and
Whites}\\
Kleindorfer, PI (04/01/15 - 03/31/20)\\
Tracking of population-based stroke incidence in the Greater Cincinnati
and Northern Kentucky region, with special emphasis on stroke in the
young and stroke recurrence.\\
Role: Biostatistician

\bigskip

\subsection*{Recently Completed Research Support}

\bigskip

\textbf{Internal Processes and Methods Award - Center for Clinical \&
  Translational Science \& Training}\\
\emph{Using Machine Learning to Supplement Electronic Health Record
  databases with Individual Socioeconomic Status}\\
Brokamp, PI (9/1/17 - 6/30/19)\\
Retrospective epidemiological studies are often created using electronic
health record databases. Although these records are ``wide'', they are
not ``deep'' with respect to individual level demographic data. We
propose a novel machine learning based approach that uses open city and
auditor databases to predict individual level income and family
socioeconomic status. This will solve the urgent problem of
unconfounding for individual SES in the execution of EHR based
research.\\
Role: PI

\bigskip

\textbf{Internal Arnold W. Strauss Fellowship Award - Cincinnati
Children's Hospital}\\
\emph{Assessing Exposure to Air Pollution Across Time and Space}\\
Brokamp, PI (7/1/16 - 6/30/17)\\
The primary objective of this award is to combine satellite-based
measurements, land use characteristics, and meteorologic data to create
a hybrid spatiotemporal model for ground level exposure to particulate
matter using exact addresses and dates.\\
Role: PI

\bigskip

\textbf{Internal Processes and Methods Award - Center for Clinical \&
Translational Science \& Training}\\
\emph{Validating a Geocoding Approach for Multi Site Studies}\\
Brokamp, PI (1/24/17 - 6/30/17)\\
The primary objective of this award is to compare the geocoding
(assigning latitude and longitude coordinates to addresses) accuracy of
our software DeGAUSS (DEcentralized Geomarker Assessment for mUlti Site
Studies) to with other common geocoding software. Furthermore, each
method will be evaluated based on it ability to correctly estimate
environmental exposures and community-level characteristics.\\
Role: PI

\bigskip

\end{document}
%!TEX TS-program = xelatex
\documentclass{nihbiosketch}

%------------------------------------------------------------------------------

\name{Brokamp, Cole}
\eracommons{brokampr}
\position{Associate Professor}

\begin{document}

\begin{education}
University of Cincinnati & B.S & 06/2010 & Biomedical Engineering \\
University of Cincinnati & Ph.D. & 04/2016 & Biostatistics and Bioinformatics \\
Cincinnati Children's Hospital Medical Center & Postdoctoral Research Fellow & 10/2017 & Biostatistics and Epidemiology \\
\end{education}

\section{Personal Statement}

\begin{statement}

  As a biostatistician, epidemiologist, and geospatial data scientist, I have specialized myself in the areas of informatics and machine learning with applications to large environmental and health outcome datasets. Recent democratization of “big spatial data” and advances in geoinformatics have allowed unprecedented access to environmental and socioeconomic characteristics that vary highly with respect to both time and space. More precise environmental features require more complex modeling and I have dedicated my career to furthering exposure science methodology in order to bring more precise exposure assessment tools to environmental and population health studies. This includes high resolution spatiotemporal exposure assessment models for fine particulate matter as well as a longitudinal measure of material community deprivation. Leveraging these models, I've also lead epidemiologic studies demonstrating the roles of air pollution, greenspace, and poverty on psychiatric and neurobehavioral child health outcomes. Furthermore, I have developed a novel approach and accompanying software package called DeGAUSS which allows for user-friendly attachment of geospatial variables to existing research cohorts while mitigating key privacy challenges. I am the founding director of the Geospatial Research Accelerator for Precision Population Health (GRAPPH), which is a shared facility at Cincinnati Children's Hospital Medical Center that works to develop and democratize geospatial data and methodologies across the institution. 

I look forward to contributing to \textit{Szczesniak\_HEAL: Genome-sociome informed risk (G-SIR) risk prediction tools for enhanced clinical management and promotion of health equity across the lifespan (HEAL)} as a project co-investigator helping to lead the geomarker assessment and algorithm fairness in precision public health.  Specifically, I recently published a racial algorithmic fairness evaluation of a widely used pediatric asthma prediction algorithm.  Additionally, I have studied the impacts of daily fluctuations in air pollution and psychiatric exacerbations.  Most importantly, I have expertise in creating geomarker assessment pipelines and have experience doing so with Dr. Szczesniak within the context of Cystic Fibrosis Foundation funding and goals.


\begin{enumerate}

  \item \textbf{Cole Brokamp}, Jeffrey R. Strawn, Andrew F. Beck, Pat Ryan.
    Pediatric Psychiatric Emergency Department Utilization and Fine
    Particulate Matter: A Case-Crossover Study. \textit{Environmental Health
      Perspectives}. 2019.

  \item \textbf{Cole Brokamp}, Andrew F. Beck, Neera K. Goyal, Patrick Ryan,
    James M. Greenberg, Eric S. Hall. Material Community Deprivation and
    Hospital Utilization During the First Year of Life: An Urban
    Population-Based Cohort Study. \textit{Annals of Epidemiology}. 30. 37-43.
    2019.

  \item \textbf{Cole Brokamp}. A High Resolution Spatiotemporal Fine Particulate Matter Exposure Assessment Model for the contiguous United States. \textit{Environmental Advances}. 7:100155. 2022.

  \item \textbf{Cole Brokamp}, Chris Wolfe, Todd Lingren, John Harley, Patrick Ryan. Decentralized and Reproducible Geocoding and Characterization of Community and Environmental Exposures for Multi-Site Studies. \textit{Journal of American Medical Informatics Association.} 25(3). 309-314. 2018.

%%   \item Christopher Wolfe, Teresa Pestian, Emrah Gecili, Weiji Su, Ruth H. Keogh, John P. Pestian, Michael Seid, Peter J. Diggle, Assem Ziady, John P. Clancy, Daniel H. Grossoehme, Rhonda D. Szczesniak, \textbf{Cole Brokamp}. Cystic Fibrosis Point of Personalized Detection (CFPOPD): An Interactive Web Application. \emph{JMIR Med Inform}. 8(12):e23530. 2020.

%%   \item Rhonda D. Szczesniak, Dan Li, Weiji Su, \textbf{Cole Brokamp}, John Pestian, Michael Seid, John P. Clancy. Phenotypes of Rapid Cystic Fibrosis Lung Disease Progression during Adolescence and Young Adulthood. \textit{American Journal of Respiratory And Critical Care Medicine}. 196(4). 471-478. 2017.

\end{enumerate}

\end{statement}

%------------------------------------------------------------------------------
\section{Positions, Scientific Appointments, and Honors}

\subsection*{Positions and Scientific Appointments}

\begin{datetbl}
2022-- & Associate Professor of Pediatrics, the University of Cincinnati Department of Pediatrics and Cincinnati Children’s Hospital Medical Center Division of Biostatistics \& Epidemiology\\
2017--2022 & Assistant Professor of Pediatrics, the University of Cincinnati Department of Pediatrics and Cincinnati Children’s Hospital Medical Center Division of Biostatistics \& Epidemiology\\
2016--2017 & Research Fellow, Cincinnati Children's Hospital Medical Center Division of Biostatistics \& Epidemiology\\
2012--2016 & Research Associate, Department of Environmental Health, University of Cincinnati \\	
\end{datetbl}

\subsection*{Honors}

\begin{datetbl}
2020 & CCHMC Epidemiology \& Biostatistics Top Publication \\
2017 & CCHMC Epidemiology \& Biostatistics Top Publication and Top Research Achievement\\
2016 & CCHMC Arnold W. Strauss Fellowship Award\\
2016 & CCHMC Division of Biostatistics \& Epidemiology Travel Award\\
2010 & B.S. awarded with Distinguished Honors, University of Cincinnati\\
\end{datetbl}

%------------------------------------------------------------------------------

\section{Contributions to Science}

\begin{enumerate}

\item \textbf{Spatiotemporal exposure assessment methods and machine learning models}

  My early career was spent developing spatiotemporal exposure assessment models for
  environmental pollutants and community characteristics based on machine learning
  techniques.  This work includes the first machine
  learning or ensemble model used to assess exposure to elemental components of
  particulate matter. Recent introduction of remote sensing satellite data has
  allowed for extension of the land use random forest model to produce daily
  estimates of air pollution back to 2000 at a resolution of 1 x 1 km. Additionally,
  I have built a validated community material deprivation index that has been used and cited
  by over 75 different published scientific studies.

\begin{enumerate}

  \item \textbf{Cole Brokamp}. A High Resolution Spatiotemporal Fine Particulate Matter Exposure Assessment Model for the contiguous United States. \textit{Environmental Advances}. 7:100155. 2022.

  \item \textbf{Cole Brokamp}, Roman Jandarov, Monir Hossain, Patrick Ryan. Predicting Daily Urban Fine Particulate Matter Concentrations Using Random Forest. \textit{Environmental Science \& Technology}. 52 (7). 4173-4179. 2018.

  \item \textbf{Cole Brokamp}, Roman Jandarov, MB Rao, Grace LeMasters, Patrick Ryan. Exposure assessment models for elemental components of particulate matter in an urban environment: A comparison of regression and random forest approaches. \textit{Atmospheric Environment}. 151. 1-11. 2017.

%%   \item	Stephen Trinidad, \textbf{Cole Brokamp}, Andres Mor Huertas, Andrew Beck, Carley Riley, Erika Rasnick, Richard Falcone, Meera Kotagal. Use of Area Based Socioeconomic Deprivation Indices: A Scoping Review and Qualitative Analysis. \textit{Health Affairs}. In Press. 2022.

%%   \item \textbf{Cole Brokamp}, Grace LeMasters, Patrick Ryan. Residential
%%     mobility impacts exposure assessment and community socioeconomic
%%     characteristics in longitudinal epidemiology studies. \emph{Journal of
%%       Exposure Science and Environmental Epidemiology}. 26(4). 428-34. 2016.

  \item \textbf{Cole Brokamp}, Andrew F. Beck, Neera K. Goyal, Patrick Ryan,
    James M. Greenberg, Eric S. Hall. Material Community Deprivation and
    Hospital Utilization During the First Year of Life: An Urban
    Population-Based Cohort Study. \textit{Annals of Epidemiology}. 30. 37-43.
    2019.

\end{enumerate}

\item \textbf{Built Environment and Pediatric Psychiatric Disorders}

  Building on advanced exposure assessment has allowed me to lead epidemiological studies on the impacts of the built environment (e.g., fine particulate matter, greenspace, community deprivation) on psychiatric and neurobehavioral pediatric health outcomes. I lead the first study to associate fine particulate matter with psychiatric outcomes in children and adolescents, using both electronic health record studies, as well as smaller, longitudinal panel studies.

  \begin{enumerate}

  \item Andrew Vancil, Jeffrey R Strawn, Erika Rasnick, Amir Levine,
    Heidi K Schroeder, Ashley M Specht, Ashley L Turner, Patrick H. Ryan,
    \textbf{Cole Brokamp}. Pediatric Anxiety and Daily Fine Particulate Matter:
    A Longitudinal Study. Psychiatry Research Communications. In Press. 2022.

  \item \textbf{Cole Brokamp}, Jeffrey R. Strawn, Andrew F. Beck, Pat Ryan.
    Pediatric Psychiatric Emergency Department Utilization and Fine
    Particulate Matter: A Case-Crossover Study. \textit{Environmental Health
      Perspectives}. 2019.
 
  \item Erika Rasnick, Patrick H. Ryan, A. John Bailer, Thomas Fisher, Patrick
    J. Parsons, Kimberly Yolton, Nicholas C. Newman, Bruce P. Lanphear,
    \textbf{Cole Brokamp}. Identifying Sensitive Windows of Airborne Lead
    Exposure Associated with Behavioral Outcomes at Age 12.
    \textit{Environmental Epidemiology}. 5(2):e144. 2021.

  \item Juliana Madzia, Patrick Ryan, Kimberly Yolton, Zana Percy, Nick Newman, Grace
    LeMasters, \textbf{Cole Brokamp}. Residential Greenspace Is Associated with Childhood
    Behavioral Outcomes. \textit{Journal of Pediatrics}. 30. 37-43. 2019.

%%   \item Clara Zundel, Patrick Ryan, \textbf{Cole Brokamp}, Autumn Heeter,
%%     Yaoxian Huang, Jeffrey Strawn, Hilary Marusak. Air Pollution, Depressive
%%     and Anxiety Disorders, and Brain Effects: A Systematic Review.
%%     NeuroToxicology. In Press. 2022.

  \end{enumerate}

\item \textbf{Privacy-based Methods and Software for Geocoding and Geomarker Assessment}

  I have developed a novel approach and accompanying software package called DeGAUSS
  which overcomes multiple privacy-related challenges in the use of address data in
  multi-site studies and also serves as a more general reproducible and scalable
  research tool for geocoding and geomarker assessment. This approach is currently being
  implemented in a wide variety of national environmental health studies. Extending this
  approach into a scalable and sustainable framework for automated integration of
  disparate and heterogeneous geomarkers via spatiotemporal location has reduced
  the need for manual data curation and specialized expertise required
  to utilize them within biomedical research studies.

\begin{enumerate}

  \item Erika Rasnick, Patrick Ryan, Jeff Blossom, Heike Luttmann-Gibson, Nathan Lothrop, 
    Rima Habre, Diane R Gold, Andrew Vancil, Joel Schwartz, James E Gern, \textbf{Cole Brokamp}.
  High Resolution and Spatiotemporal Place-Based Computable Exposures at Scale. \textit{AMIA Summits
    on Translational Science Proceedings}. In Press. 2023.
	
\item Patrick H. Ryan, \textbf{Cole Brokamp}, Jeff Blossom, Nathan Lothrop,
  Rachel L. Miller, Paloma I. Beamer, Cynthia M. Visness, Antonella
  Zanobetti, Howard Andrews, Leonard B. Bacharier, Tina Hartert, Christine
  C. Johnson, Dennis Ownby, Robert F. Lemanske, Jr., Heike Gibson, Weeberb
  Requia, Brent Coull, Edward M. Zoratti, Anne L. Wright, Fernando D.
  Martinez, Christine M. Seroogy, James E. Gern, Diane R. Gold, on behalf
  of the CREW Consortium. A Distributed Geospatial Approach to Describe
  Community Characteristics for Multi-Site Studies. \textit{Journal of
    Clinical and Translational Science}. 5:e86, 1-8. 2021.

  \item \textbf{Cole Brokamp}, Chris Wolfe, Todd Lingren, John Harley, Patrick Ryan. Decentralized and Reproducible Geocoding and Characterization of Community and Environmental Exposures for Multi-Site Studies. \textit{Journal of American Medical Informatics Association.} 25(3). 309-314. 2018.
	
  \item \textbf{Cole Brokamp}. DeGAUSS: Decentralized Geomarker Assessment for Multi-Site Studies. \textit{Journal of Open Source Software}. 2018. 

\end{enumerate}

\item \textbf{Pediatric Health Disparities}

  I have also contributed to several studies on the disparities of health outcomes within children and the contribution of the place-based and social determinants of health to these disparities in order to identify root causes and meaningful solutions.

\begin{enumerate}

  \item Joanna Thomson, Breann Butts, Saige Camara, Erika Rasnick, \textbf{Cole Brokamp}, Caroline Heyd, Rebecca Steuart, Scott Callahan, Stuart Taylor, Andrew Beck. Neighborhood Socioeconomic Deprivation and Health Care Utilization of Medically Complex Children. \textit{Pediatrics}. e2021052592. 2022.

  \item Erica Andrist, \textbf{Cole Brokamp}, Stuart Taylor, Carley Riley,
  Andrew Beck. Neighborhood Poverty and Pediatric Intensive Care Use.
  \textit{Pediatrics}. 2019.

	\item Andrew F. Beck, Carley L. Riley, Stuart Taylor, \textbf{Cole Brokamp},
    Robert S. Kahn. Toward a Culture of Health in Hospitals: Pervasive
    population disparities in inpatient bed-day rates across conditions and
    subspecialties. \textit{Health Affairs}. 37(4). 551-559. 2018.
		
	\item Lauren C. Riney, \textbf{Cole Brokamp}, Andrew F. Beck, Wendy Pomerantz,
    Hamilton Schwartz, Todd A. Florin. Emergency Medical Services Utilization is
    Associated with Community Deprivation in Children. \textit{Prehospital
      Emergency Care}. 2018.

\end{enumerate}

\item \textbf{Clinical Forecasting in Cystic Fibrosis with Geomarkers}

  Lastly, I have contributed to a research team that has recently used
  functional data analysis combined with joint modeling (FD-JM) to identify and
  predict rapid decline in lung function among patients with cystic fibrosis
  (CF) lung disease. My work in translating this predictive model into an interactive application has allowed for patients and clinicians to take advantage of it at the bedside.  Focus groups and partnerships with clinicians have allowed us to iteratively develop the application based on end-user feedback. Work with the CF Foundation Patient Registry (CFFPR) to implement these models and visualizations into clinical settings has improved prognostic care.

\begin{enumerate}

  \item Christopher Wolfe, Teresa Pestian, Emrah Gecili, Weiji Su, Ruth H. Keogh, John P. Pestian, Michael Seid, Peter J. Diggle, Assem Ziady, John P. Clancy, Daniel H. Grossoehme, Rhonda D. Szczesniak, \textbf{Cole Brokamp}. Cystic Fibrosis Point of Personalized Detection (CFPOPD): An Interactive Web Application. \emph{JMIR Med Inform}. 8(12):e23530. 2020.

  \item Rhonda Szczesniak, \textbf{Cole Brokamp}, Weiji Su, Gary L. McPhail, John Pestian, John P. Clancy. Early Detection of Rapid Cystic Fibrosis Disease Progression Tailored to Point of Care: A Proof-of-Principle Study. \textit{Healthcare Innovations and Point of Care Technologies}. (HI-POCT), 2017 IEEE. 204-207. 2017.

  \item Rhonda D. Szczesniak, Dan Li, Weiji Su, \textbf{Cole Brokamp}, John Pestian, Michael Seid, John P. Clancy. Phenotypes of Rapid Cystic Fibrosis Lung Disease Progression during Adolescence and Young Adulthood. \textit{American Journal of Respiratory And Critical Care Medicine}. 196(4). 471-478. 2017.

\item US Patent: Assem Ziady, Rhonda Szczesniak, John Clancy, \textbf{Cole Brokamp}, inventors; Cincinnati Children's Hospital Medical Center, assignee. Compositions and methods for treatment of lung function. United States patent US 10,761,099. 2020 Sep 1.

\end{enumerate}

\end{enumerate}

\subsection*{Complete List of Published Work in ORCiD:} 
\url{https://orcid.org/0000-0002-0289-3151}

\end{document}
